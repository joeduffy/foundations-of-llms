% Frontispiece: microGPT by Andrej Karpathy
% Karpathy's "Let there be..." comments are elevated to section headers,
% creating a creation narrative interspersed with readable code.
\clearpage
\thispagestyle{empty}

% Disable microtype for this section (conflicts with monospace at small sizes)
\microtypesetup{activate=false}

% Custom command for the creation-narrative headers
\newcommand{\creationheader}[1]{%
    \par\goodbreak\vspace{18pt plus 8pt minus 4pt}%
    \noindent{\fontsize{12}{14}\selectfont\itshape #1}%
    \par\nopagebreak\vspace{4pt}%
}

% Sub-header for secondary annotations
\newcommand{\creationsubheader}[1]{%
    \par\nopagebreak\vspace{2pt}%
    \noindent{\fontsize{9}{11}\selectfont\itshape #1}%
    \par\nopagebreak\vspace{4pt}%
}

% Custom verbatim environment for code blocks
\DefineVerbatimEnvironment{MicroCode}{Verbatim}{%
    fontsize=\fontsize{7.5}{9.5}\selectfont,
    formatcom=\fontfamily{lmtt}\selectfont,
    breaklines=true,
    breaksymbol={},
    breakanywheresymbolpre={},
}

%% ---------------------------------------------------------------
%%  TITLE AND INTRODUCTION
%% ---------------------------------------------------------------

\begin{center}
{\fontsize{16}{18}\selectfont\scshape The Algorithm}
\end{center}

\vspace{10pt}

\begin{center}
\begin{minipage}{0.82\textwidth}
\centering\fontsize{9.5}{12}\selectfont\itshape
``The most atomic way to train and inference a GPT\\
in pure, dependency-free Python.\\[3pt]
This file is the complete algorithm.\\
Everything else is just efficiency.''
\par\vspace{6pt}
\upshape\hfill ---Andrej Karpathy
\end{minipage}
\end{center}

\vspace{10pt}

{\fontsize{9.5}{12}\selectfont\noindent
The listing that follows is a complete, working GPT language model---tokenizer,
scalar-valued autograd engine, multi-head causal self-attention, transformer
block with RMSNorm, Adam optimizer, training loop, and autoregressive text
generation---in approximately 200 lines of dependency-free Python. Its only
mathematical primitives are addition, multiplication, exponentiation, and the
natural logarithm, composed through the chain rule. The papers in this volume
tell the story of how each of these ideas was discovered, and how to make them
work at scale.\par}

\vspace{6pt}
\begin{center}\rule{0.5\textwidth}{0.4pt}\end{center}

%% ---------------------------------------------------------------
%%  I. IMPORTS & SEED
%% ---------------------------------------------------------------

\creationheader{Let there be order among chaos.}
\begin{MicroCode}
import os  # os.path.exists
import math  # math.log, math.exp
import random  # random.seed, random.choices, random.gauss, random.shuffle

random.seed(42)
\end{MicroCode}

%% ---------------------------------------------------------------
%%  II. DATASET
%% ---------------------------------------------------------------

\creationheader{Let there be an input dataset.}
\begin{MicroCode}
if not os.path.exists('input.txt'):
    import urllib.request
    names_url = 'https://raw.githubusercontent.com/karpathy/makemore/refs/heads/master/names.txt'
    urllib.request.urlretrieve(names_url, 'input.txt')

docs = [l.strip() for l in open('input.txt').read().strip().split('\n') if l.strip()]
random.shuffle(docs)
print(f"num docs: {len(docs)}")
\end{MicroCode}

%% ---------------------------------------------------------------
%%  III. TOKENIZER
%% ---------------------------------------------------------------

\creationheader{Let there be a Tokenizer, to translate strings to discrete symbols and back.}
\begin{MicroCode}
uchars = sorted(set(''.join(docs)))  # unique chars → token ids 0..n-1
BOS = len(uchars)  # token id for the special Beginning of Sequence (BOS) token
vocab_size = len(uchars) + 1  # total number of unique tokens, +1 is for BOS
print(f"vocab size: {vocab_size}")
\end{MicroCode}

%% ---------------------------------------------------------------
%%  IV. AUTOGRAD
%% ---------------------------------------------------------------

\creationheader{Let there be Autograd, to recursively apply the chain rule through a computation graph.}
\begin{MicroCode}
class Value:
    __slots__ = ('data', 'grad', '_children', '_local_grads')

    def __init__(self, data, children=(), local_grads=()):
        self.data = data  # scalar value of this node calculated during forward pass
        self.grad = 0  # derivative of the loss w.r.t. this node, calculated in backward pass
        self._children = children  # children of this node in the computation graph
        self._local_grads = local_grads  # local derivative of this node w.r.t. its children

    def __add__(self, other):
        other = other if isinstance(other, Value) else Value(other)
        return Value(self.data + other.data, (self, other), (1, 1))

    def __mul__(self, other):
        other = other if isinstance(other, Value) else Value(other)
        return Value(self.data * other.data, (self, other), (other.data, self.data))

    def __pow__(self, other): return Value(self.data**other, (self,), (other * self.data**(other-1),))
    def log(self): return Value(math.log(self.data), (self,), (1/self.data,))
    def exp(self): return Value(math.exp(self.data), (self,), (math.exp(self.data),))
    def relu(self): return Value(max(0, self.data), (self,), (float(self.data > 0),))
    def __neg__(self): return self * -1
    def __radd__(self, other): return self + other
    def __sub__(self, other): return self + (-other)
    def __rsub__(self, other): return other + (-self)
    def __rmul__(self, other): return self * other
    def __truediv__(self, other): return self * other**-1
    def __rtruediv__(self, other): return other * self**-1

    def backward(self):
        topo = []
        visited = set()

        def build_topo(v):
            if v not in visited:
                visited.add(v)
                for child in v._children:
                    build_topo(child)
                topo.append(v)

        build_topo(self)
        self.grad = 1
        for v in reversed(topo):
            for child, local_grad in zip(v._children, v._local_grads):
                child.grad += local_grad * v.grad
\end{MicroCode}

%% ---------------------------------------------------------------
%%  V. PARAMETERS
%% ---------------------------------------------------------------

\creationheader{Initialize the parameters, to store the knowledge of the model.}
\begin{MicroCode}
n_embd = 16  # embedding dimension
n_head = 4  # number of attention heads
n_layer = 1  # number of layers
block_size = 16  # maximum sequence length
head_dim = n_embd // n_head  # dimension of each head

matrix = lambda nout, nin, std=0.08: [[Value(random.gauss(0, std)) for _ in range(nin)] for _ in range(nout)]

state_dict = {'wte': matrix(vocab_size, n_embd), 'wpe': matrix(block_size, n_embd),
              'lm_head': matrix(vocab_size, n_embd)}

for i in range(n_layer):
    state_dict[f'layer{i}.attn_wq'] = matrix(n_embd, n_embd)
    state_dict[f'layer{i}.attn_wk'] = matrix(n_embd, n_embd)
    state_dict[f'layer{i}.attn_wv'] = matrix(n_embd, n_embd)
    state_dict[f'layer{i}.attn_wo'] = matrix(n_embd, n_embd)
    state_dict[f'layer{i}.mlp_fc1'] = matrix(4 * n_embd, n_embd)
    state_dict[f'layer{i}.mlp_fc2'] = matrix(n_embd, 4 * n_embd)

params = [p for mat in state_dict.values() for row in mat for p in row]
print(f"num params: {len(params)}")
\end{MicroCode}

%% ---------------------------------------------------------------
%%  VI. MODEL ARCHITECTURE
%% ---------------------------------------------------------------

\creationheader{Define the model architecture.\\[2pt]
{\fontsize{9}{11}\selectfont Follow GPT-2, blessed among the GPTs, with minor differences:\\
layernorm\,$\rightarrow$\,rmsnorm, no biases, GeLU\,$\rightarrow$\,ReLU.}}
\begin{MicroCode}
def linear(x, w):
    return [sum(wi * xi for wi, xi in zip(wo, x)) for wo in w]

def softmax(logits):
    max_val = max(val.data for val in logits)
    exps = [(val - max_val).exp() for val in logits]
    total = sum(exps)
    return [e / total for e in exps]

def rmsnorm(x):
    ms = sum(xi * xi for xi in x) / len(x)
    scale = (ms + 1e-5) ** -0.5
    return [xi * scale for xi in x]

def gpt(token_id, pos_id, keys, values):
    tok_emb = state_dict['wte'][token_id]  # token embedding
    pos_emb = state_dict['wpe'][pos_id]  # position embedding
    x = [t + p for t, p in zip(tok_emb, pos_emb)]  # joint token and position embedding
    x = rmsnorm(x)

    for li in range(n_layer):
        # 1) Multi-head attention block
        x_residual = x
        x = rmsnorm(x)
        q = linear(x, state_dict[f'layer{li}.attn_wq'])
        k = linear(x, state_dict[f'layer{li}.attn_wk'])
        v = linear(x, state_dict[f'layer{li}.attn_wv'])
        keys[li].append(k)
        values[li].append(v)

        x_attn = []
        for h in range(n_head):
            hs = h * head_dim
            q_h = q[hs:hs+head_dim]
            k_h = [ki[hs:hs+head_dim] for ki in keys[li]]
            v_h = [vi[hs:hs+head_dim] for vi in values[li]]
            attn_logits = [sum(q_h[j] * k_h[t][j] for j in range(head_dim)) / head_dim**0.5
                           for t in range(len(k_h))]
            attn_weights = softmax(attn_logits)
            head_out = [sum(attn_weights[t] * v_h[t][j] for t in range(len(v_h)))
                        for j in range(head_dim)]
            x_attn.extend(head_out)

        x = linear(x_attn, state_dict[f'layer{li}.attn_wo'])
        x = [a + b for a, b in zip(x, x_residual)]

        # 2) MLP block
        x_residual = x
        x = rmsnorm(x)
        x = linear(x, state_dict[f'layer{li}.mlp_fc1'])
        x = [xi.relu() for xi in x]
        x = linear(x, state_dict[f'layer{li}.mlp_fc2'])
        x = [a + b for a, b in zip(x, x_residual)]

    logits = linear(x, state_dict['lm_head'])
    return logits
\end{MicroCode}

%% ---------------------------------------------------------------
%%  VII. ADAM OPTIMIZER & TRAINING
%% ---------------------------------------------------------------

\creationheader{Let there be Adam, the blessed optimizer.}
\begin{MicroCode}
learning_rate, beta1, beta2, eps_adam = 0.01, 0.85, 0.99, 1e-8
m = [0.0] * len(params)  # first moment buffer
v = [0.0] * len(params)  # second moment buffer

num_steps = 1000  # number of training steps

for step in range(num_steps):
    doc = docs[step % len(docs)]
    tokens = [BOS] + [uchars.index(ch) for ch in doc] + [BOS]
    n = min(block_size, len(tokens) - 1)

    # Forward the token sequence through the model
    keys, values = [[] for _ in range(n_layer)], [[] for _ in range(n_layer)]
    losses = []

    for pos_id in range(n):
        token_id, target_id = tokens[pos_id], tokens[pos_id + 1]
        logits = gpt(token_id, pos_id, keys, values)
        probs = softmax(logits)
        loss_t = -probs[target_id].log()
        losses.append(loss_t)

    loss = (1 / n) * sum(losses)  # May yours be low.

    # Backward the loss
    loss.backward()

    # Adam update
    lr_t = learning_rate * (1 - step / num_steps)  # linear learning rate decay
    for i, p in enumerate(params):
        m[i] = beta1 * m[i] + (1 - beta1) * p.grad
        v[i] = beta2 * v[i] + (1 - beta2) * p.grad ** 2
        m_hat = m[i] / (1 - beta1 ** (step + 1))
        v_hat = v[i] / (1 - beta2 ** (step + 1))
        p.data -= lr_t * m_hat / (v_hat ** 0.5 + eps_adam)
        p.grad = 0

    print(f"step {step+1:4d} / {num_steps:4d} | loss {loss.data:.4f}")
\end{MicroCode}

%% ---------------------------------------------------------------
%%  VIII. INFERENCE
%% ---------------------------------------------------------------

\creationheader{May the model babble back to us.}
\begin{MicroCode}
temperature = 0.5  # in (0, 1], control the "creativity" of generated text

print("\n--- inference (new, hallucinated names) ---")

for sample_idx in range(20):
    keys, values = [[] for _ in range(n_layer)], [[] for _ in range(n_layer)]
    token_id = BOS
    sample = []

    for pos_id in range(block_size):
        logits = gpt(token_id, pos_id, keys, values)
        probs = softmax([l / temperature for l in logits])
        token_id = random.choices(range(vocab_size), weights=[p.data for p in probs])[0]

        if token_id == BOS:
            break

        sample.append(uchars[token_id])

    print(f"sample {sample_idx+1:2d}: {''.join(sample)}")
\end{MicroCode}

%% ---------------------------------------------------------------
%%  ATTRIBUTION
%% ---------------------------------------------------------------

\vspace{6pt}
\begin{center}\rule{0.5\textwidth}{0.4pt}\end{center}
\vspace{4pt}

\begin{center}
\fontsize{8}{10}\selectfont
Andrej Karpathy, \textit{microGPT} (2026).\quad
\texttt{karpathy.ai/microgpt.html}\quad\textbar\quad
\texttt{gist.github.com/karpathy/8627fe009c40f57531cb18360106ce95}
\end{center}

%% ---------------------------------------------------------------
%%  CLEANUP
%% ---------------------------------------------------------------

\microtypesetup{activate=true}
\clearpage
