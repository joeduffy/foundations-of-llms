% Paper Summary: Foundation Agents: A Comprehensive Survey (Liu et al., 2025)

\begin{papersummary}{2025}{Foundation Agents: A Comprehensive Survey}{Yuxin Liu, Jie Chen, Xinyu Wang, Zeming Chen, Yichen Zhou, Shubin Huang, et al.}{This comprehensive survey organizes the rapidly evolving field of LLM-based agents, providing taxonomy, benchmarks, and research directions.}

\summaryheading{Key Ideas}
The survey systematically organizes work on foundation agents---AI systems that use large pretrained models for autonomous task completion. It covers agent architectures (reasoning, planning, memory), interaction paradigms (tool use, environment interaction, multi-agent systems), and application domains (software engineering, scientific discovery, embodied AI). The paper identifies key challenges: long-horizon planning, grounding in dynamic environments, safe and aligned behavior, and evaluation methodology. It provides a comprehensive taxonomy that helps researchers position their work and identify open problems.

\summaryheading{Follow-on Works}
As a recent survey, this paper synthesizes rather than generates follow-on work. However, it identifies research frontiers that will likely see significant activity: compositional generalization in agents, learning from interaction, multi-agent coordination, and safety guarantees for agentic systems.

\summaryheading{Lasting Contributions}
This survey provides an essential reference for the rapidly growing field of LLM agents. By establishing common terminology and taxonomy, it enables clearer communication across research groups. The identification of key challenges---particularly around long-horizon reasoning, grounding, and safety---helps focus the research community on important problems. As agents become more capable and widely deployed, having a shared understanding of the landscape becomes increasingly valuable for coordinated progress.

\end{papersummary}
