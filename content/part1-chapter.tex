Modern language models derive from mathematical foundations established between 1943 and 1990. McCulloch and Pitts formalized artificial neuron concepts, establishing computational frameworks that underlie current neural architectures.

This period introduced three core concepts: artificial neurons, perceptron learning algorithms, and backpropagation. These developments emerged within early computing research and cybernetics, establishing mathematical tools for machine learning systems.

\section*{Key Advances}

This era introduced the core mathematical abstractions that underpin all neural networks:

\textbf{Artificial Neurons (1943):} McCulloch and Pitts demonstrated that networks of simple threshold units could perform arbitrary logical computations, establishing the theoretical foundation for neural computation.

\textbf{Learning Algorithms (1958):} Rosenblatt's perceptron introduced the first systematic approach to training neural networks, showing how machines could learn from examples.

\textbf{Associative Memory (1982):} Hopfield networks demonstrated how neural systems could store and retrieve patterns, introducing energy-based models and the concept of attractors in neural dynamics.

\textbf{Backpropagation (1986):} Rumelhart, Hinton, and Williams solved the credit assignment problem in multilayer networks, enabling the training of deep architectures that would later prove essential for language modeling.

\section*{Papers in This Section}

\begin{enumerate}
\item \textbf{\hyperref[paper:mcculloch-pitts-1943]{[1943] A Logical Calculus of the Ideas Immanent in Nervous Activity (McCulloch \& Pitts)}} -- Established the mathematical foundation of artificial neurons and their computational capabilities.

\item \textbf{\hyperref[paper:rosenblatt-1958]{[1958] The Perceptron: A Probabilistic Model for Information Storage (Rosenblatt)}} -- Introduced the perceptron learning algorithm and demonstrated machine learning from examples.

\item \textbf{\hyperref[paper:hopfield-1982]{[1982] Neural Networks and Physical Systems with Emergent Collective Computational Abilities (Hopfield)}} -- Developed associative memory networks and energy-based learning, introducing concepts of neural dynamics.

\item \textbf{\hyperref[paper:rumelhart-hinton-williams-1986]{[1986] Learning Representations by Back-Propagating Errors (Rumelhart, Hinton \& Williams)}} -- Solved the credit assignment problem with backpropagation, enabling training of deep networks.
\end{enumerate}

These papers established the mathematical tools and learning principles that would prove essential for the development of transformer architectures and large language models decades later. The concepts of distributed representation, gradient-based learning, and hierarchical feature extraction introduced here remain fundamental to modern AI systems.

% Papers follow directly after the introduction

\phantomsection
\label{paper:mcculloch-pitts-1943}
\addcontentsline{toc}{subsection}{[1943] A Logical Calculus of the Ideas Immanent in Nervous Activity (McCulloch \& Pitts)}
\includepdf[pages=-,pagecommand={\thispagestyle{empty}}]{pdfs/mcculloch-pitts-1943.pdf}

\phantomsection
\label{paper:rosenblatt-1958}
\addcontentsline{toc}{subsection}{[1958] The Perceptron: A Probabilistic Model for Information Storage (Rosenblatt)}
\includepdf[pages=-,pagecommand={\thispagestyle{empty}}]{pdfs/rosenblatt-1958.pdf}

\phantomsection
\label{paper:hopfield-1982}
\addcontentsline{toc}{subsection}{[1982] Neural Networks and Physical Systems with Emergent Collective Computational Abilities (Hopfield)}
\includepdf[pages=-,pagecommand={\thispagestyle{empty}}]{pdfs/hopfield-1982.pdf}

\phantomsection
\label{paper:rumelhart-hinton-williams-1986}
\addcontentsline{toc}{subsection}{[1986] Learning Representations by Back-Propagating Errors (Rumelhart, Hinton \& Williams)}
\includepdf[pages=-,pagecommand={\thispagestyle{empty}}]{pdfs/rumelhart-hinton-williams-1986.pdf}