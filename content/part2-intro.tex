\chapter*{Introduction to Part II}
\addcontentsline{toc}{chapter}{Introduction to Part II}

The period from 1990 to 2013 marked the emergence of sequence modeling and distributed word representations that would prove essential for large language models. This era witnessed the development of recurrent neural networks capable of processing sequential data and the first neural approaches to language modeling.

The key insight driving this period was that language could be modeled as sequences of symbols, with neural networks learning statistical patterns from large corpora. This represented a fundamental shift from rule-based natural language processing toward data-driven approaches that could capture the statistical regularities of human language.

\section*{Key Advances}

\textbf{Long Short-Term Memory (1997):} Hochreiter and Schmidhuber solved the vanishing gradient problem in recurrent networks, enabling the modeling of long-term dependencies essential for language understanding.

\textbf{Neural Language Models (2003):} Bengio et al. demonstrated that neural networks could learn distributed representations of words while modeling language probability distributions, establishing the foundation for modern language modeling.

\textbf{Word Embeddings (2013):} Mikolov et al. showed that high-quality word representations could be learned efficiently from large corpora, revolutionizing how machines represent and process language.

\textbf{Sequence Generation (2013):} Graves demonstrated that recurrent networks could generate coherent sequences, including realistic handwriting, pointing toward the generative capabilities of modern language models.

\section*{Papers in This Section}

\begin{enumerate}
\item \textbf{Hochreiter \& Schmidhuber (1997)} -- Introduced LSTM networks, solving the vanishing gradient problem for sequence modeling.

\item \textbf{Bengio et al. (2003)} -- Established neural language modeling and distributed word representations.

\item \textbf{Glorot \& Bengio (2010)} -- Analyzed training dynamics and proposed normalized initialization schemes that enabled stable training of deep networks.

\item \textbf{Collobert et al. (2011)} -- Demonstrated that neural networks could learn useful NLP representations from unlabeled data, reducing reliance on feature engineering.

\item \textbf{Mikolov et al. (2013)} -- Developed efficient word embedding techniques that became foundational for all language models.

\item \textbf{Kingma \& Welling (2013)} -- Introduced variational autoencoders, providing a principled framework for learning latent representations.

\item \textbf{Graves (2013)} -- Demonstrated neural sequence generation capabilities, presaging modern generative language models.
\end{enumerate}