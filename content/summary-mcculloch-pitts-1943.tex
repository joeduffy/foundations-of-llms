% Paper Summary: A Logical Calculus of the Ideas Immanent in Nervous Activity (McCulloch & Pitts, 1943)

\begin{papersummary}{1943}{A Logical Calculus of the Ideas Immanent in Nervous Activity}{Warren S. McCulloch, Walter Pitts}{This foundational paper established that networks of binary threshold units can perform arbitrary logical computations, providing the mathematical basis for artificial neural networks.}

\summaryheading{Key Ideas}
The paper introduced neurons as threshold functions with weighted inputs, demonstrating how excitatory and inhibitory connections enable complex computation through simple primitives. By proving that such networks are Turing-complete, McCulloch and Pitts established that any computable function could be implemented by neural networks. This bridged neuroscience with mathematical logic, proposing that mental processes could be understood through formal computational systems rather than remaining mystical phenomena.

\summaryheading{Follow-on Works}
Rosenblatt's Perceptron (\hyperref[paper:rosenblatt-1958]{1958}) built on threshold units by introducing learning algorithms. Hopfield networks (\hyperref[paper:hopfield-1982]{1982}) explored recurrent connections and energy dynamics. Backpropagation (\hyperref[paper:rumelhart-hinton-williams-1986]{1986}) generalized these binary units to continuous activations with differentiable functions, enabling gradient-based training of deep networks.

\summaryheading{Lasting Contributions}
While modern networks use continuous activations rather than binary thresholds, McCulloch-Pitts neurons established the fundamental computational model underlying all neural architectures. Every neuron in contemporary LLMs—from embedding layers through attention mechanisms to output projections—operates on the principle of weighted summation and nonlinear activation introduced here. The insight that complex intelligence emerges from networks of simple computational units remains central to understanding how language models process and generate text.

\end{papersummary}
